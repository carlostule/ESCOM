\documentclass{article}

\usepackage[spanish]{babel}
\usepackage[utf8]{inputenc} % Para usar tildes Unicode
\usepackage[top=0.1cm, bottom=0.5cm, left=2cm, right=2cm]{geometry}
\usepackage[colorlinks=true, urlcolor=blue]{hyperref}

\setlength{\parindent}{1cm}

\title{Dra. María Esther Orozco}
\author{Ortega Ortu\~no Eder}
\date{} % Para evitar que salga la fecha al llamar a 'maketitle'

\begin{document}
	\pagenumbering{gobble} % Ocultar pagenumber
	\maketitle
	\normalsize{
La Dra. María Esther Orozco Orozco es una investigadora mexicana que ha ocupado el cargo de rectora en la Universidad Autónoma de la Ciudad de México, aunque en la parte académica tiene estudios de posgrado en Biología Celular. En su juventud fue profesora en la Escuela del Estado de Chihuahua para posteriormente realizar sus estudios de licenciatura en la Universidad Autónoma de Chihuahua.
\\

Posterior a ello realizó sus estudios de posgrado de Biología Celular en el CINVESTAV-IPN para más tarde impartir clases en el departamento de Genética y Biología Molecular; eventualmente se desplazó al departamento de Infectómica y Patogénesis Molecular. En aquellos departamentos ha trabajado con otros investigadores y alumnos con proyectos y tesis de maestría y doctorado, de los cuales muchos han sido profesores en el IPN, UNAM, UAM, UACM e incluso universidades en el extranjero.
\\

Ha sido parte de organizaciones y grupos de investigadores que han trabajado en proyectos como el Programa de Biomedicina Molecular que más tarde fue aceptado como un departamento en el mismo CINVESTAV-IPN, así como del mismo surgió un programa de posgrado en Biomedicina Molecular en la Escuela Nacional de Medicina y Homeopatía (ENMyH-IPN).
\\

En cuanto a sus investigaciones, la Dra. Esther Orozco se ha enfocado en el estudio de genes y proteínas así como la resistencia a fármacos en genes particulares aplicando biología y genética molecular. Basándose en lo anterior ha publicado diferentes artículos en revistas de divulgación científica de alcance internacional; además de participar en programas científicos del canal 11 perteneciente al Instituto Politécnico Nacional.
\\

La investigadora ha recibido diferentes reconocimientos y medallas por parte de la Secretaría de Salud, UNESCO y UNAM, específicamente de programas científicos. Sin embargo también se ha visto envuelta en escándalos de carácter público con medios de comunicación de alcance masivo los cuales aseguran que ella no tenía su cédula profesional y que su reconocimiento ha sido gracias a personas involucradas en fraudes con la Dra. María Esther Orozco.
\\

Finalmente la doctora también ha participado en el ámbito social al ocupar cargos como servidora pública o como fundadora y directora de programas sociales por parte del gobierno del Distrito Federal que tienen como objetivo ampliar el campo científico con apoyo de instituciones educativas como el IPN, UNAM y UACM.
\\
}

\vspace{1cm}

\section*{Bibliograf\'ia}

\noindent \url{http://infectomica.cinvestav.mx/PersonalAcad%C3%A9mico/DraOrozcoOrozcoMaEsther.aspx}
\\
\noindent \url{http://www.foroconsultivo.org.mx/comisiones_sni_2014/cv/a2_maria_esther_orozco_orozco.pdf}
\\
\noindent \url{http://www.estherorozco.net/}
\\

\large{\hfill \textbf{Hecho en } \LaTeX - \url{multiaportes.com}}

\end{document}