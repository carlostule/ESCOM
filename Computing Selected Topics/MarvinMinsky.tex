\documentclass{article}

\usepackage[spanish]{babel}
\usepackage[utf8]{inputenc} % Para usar tildes Unicode
\usepackage[top=0.1cm, bottom=0.5cm, left=2cm, right=2cm]{geometry}
\usepackage[colorlinks=true, urlcolor=blue]{hyperref}

\setlength{\parindent}{1cm}

\title{Marvin Minsky}
\author{Ortega Ortu\~no Eder}
\date{} % Para evitar que salga la fecha al llamar a 'maketitle'

\begin{document}
	\pagenumbering{gobble} % Ocultar pagenumber
	\maketitle
	\normalsize{
Marvin Lee Minsky, considerado como el padre de la inteligencia artificial, es un investigador nacido en el año de 1927 en Nueva York; concluyó sus estudios de posgrado en Harvard y Princeton para posteriormente dedicarse a los campos de las redes neuronales, teoría de autómatas, psicología cognitiva, representación del conocimiento e inteligencia artificial.
\\

Un logro destacable del investigador estadounidense es fundar el Laboratorio de Inteligencia Artificial del Instituto Tecnológico de Massachussetts en el año de 1959 con el propósito de realizar investigaciones en ese campo a nivel mundial.
\\

Una situación particular en la vida académica de Marvin Minsky fue lograr que otros investigadores reconocidos como John Von Neumann, Claude Shannon y Norbert Wiener aceptaran su admisión al grupo de Junior Fellows (un programa académico del MIT) tras haber concluído sus estudios de posgrado. Allí realizó investigaciones con microscopios, para más tarde enfocarse al área de la biología y ciencia de los materiales.
\\

Uno de sus trabajos en el área de la Inteligencia Artificial fue mediante una investigación plasmada en artículo titulado "Los pasos hacia la Inteligencia Artificial" en el cual hace mención de la manipulación de símbolos mediante técnicas como búsquedas heurísticas, reconocimiento de patrones y aprendizaje e inducción.
\\

Algunos años después, el científico en cuestión comenzó a trabajar en una serie de dispositivos conocidos como perceptrones utilizó para estudiar el comportamiento neuronal, así como lo que estos dispositivos podían y no podían realizar. Eventualmente este tipo de investigaciones le llevó a escribir algunos artículos enfocados a los desarrollos contemporáneos en esta misma rama de la ciencia.
\\

En el año 2006, el doctor Minsky publicó un libro titulado "La emoción de las máquinas" donde habla sobre la conciencia, emociones y el sentido común utilizando diferentes puntos de vista así como las reacciones instintivas y las reacciones aprendidas.

}

\vspace{1cm}

\section*{Bibliograf\'ia}

\noindent \url{http://www.computerhistory.org/fellowawards/hall/bios/Marvin,Minsky/}
\\
\noindent \url{http://amturing.acm.org/award_winners/minsky_7440781.cfm}
\\

\large{\hfill \textbf{Hecho en } \LaTeX - \url{multiaportes.com}}

\end{document}