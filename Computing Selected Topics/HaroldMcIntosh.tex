\documentclass{article}

\usepackage[spanish]{babel}
\usepackage[utf8]{inputenc} % Para usar tildes Unicode
\usepackage[top=0.1cm, bottom=0.5cm, left=2cm, right=2cm]{geometry}
\usepackage[colorlinks=true, urlcolor=blue]{hyperref}

\setlength{\parindent}{1cm}

\title{Harold V. McIntosh}
\author{Ortega Ortu\~no Eder}
\date{} % Para evitar que salga la fecha al llamar a 'maketitle'

\begin{document}
	\pagenumbering{gobble} % Ocultar pagenumber
	\maketitle
	\normalsize{
El científico Harold V. McIntosh tiene estudios de posgrado en Física, Química y Matemáticas que obtuvo en la década de 1940 a 1970; reconocido por tener una grandiosa capacidad de hacer aprender a la gente como al premio Nobel de Física Sheldon L. Glashow que inclusive lo vivió en persona y asegura haber aprendido demasiado sobre las temáticas que en su momento trataron.
\\

En particular este investigador fue parte del CINVESTAV-IPN y CENAC; director de la tesis de Adolfo Guzmán Arenas, específiamente de un compilador para un lenguaje idea del mismo McIntosh conocido como CONVERT y que está basado en LISP. Además de ello también creó un compilador de otro lenguaje conocido como Regular Expression Compiler (REC) mientras dirigía el Departamento de Programación del Centro de Cálculo Electrónico UNAM.
\\

A su vez, ha impartido clases en la Escuela Superior de Física y Matemáticas así como haber hecho colaboraciones en el Instituto Nacional de Tecnología Nuclear; tras la fundación de la carrera de Licenciatura en Computación en la Escuela de Ciencias Físico-Matemáticas en la Universidad Autónoma de Puebla, fue invitado a colaborar y eventualmente logró que fuera una de las mejores carreras enfocadas a las matemáticas y la computación.
\\

Mientras tanto, la empresa Intel lanzó una familia de microprocesadores que fueron el motivo ideal para fundar el Departamento de Aplicación de Microcomputadoras en el Instituto de Ciencias de la misma Universidad, con la que posteriormente comenzaron a realizar trabajos de investigación enfocadas a la computación.
\\

Entre otras cosas, el investigador Harold V. McIntosh ha realizado trabajos sobresalientes en el departamento previamente mencionado como una computadora personal CP-UAP, una versión del lenguaje REC adaptado a la PDP-8, posteriormente una versión nueva de CONVERT e incluso un proyecto para el aprendizaje de teoría de autómatas con las características de simular autómatas de pila y Turing e inclusive ejecutar algunos programas escritos en el lenguaje de programación LISP.
}

\vspace{1cm}

\section*{Bibliograf\'ia}

\noindent \url{http://delta.cs.cinvestav.mx/~mcintosh/comun/gto91/node2.html}
\\
\noindent \url{http://www.fgalindosoria.com/informaticos/fundamentales/Harold_V_McIntosh/}
\\

\large{\hfill \textbf{Hecho en } \LaTeX - \url{multiaportes.com}}

\end{document}